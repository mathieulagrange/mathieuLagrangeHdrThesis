\documentclass[a4paper, french, 10pt]{book}
\usepackage[utf8]{inputenc}
%\usepackage[T1]{fontenc}
%\usepackage[applemac]{inputenc}
\usepackage[french]{babel}
\usepackage{euscript,amsmath,amssymb,amsfonts,amsthm,epsfig,subfigure,color}
\usepackage{latexsym}
\usepackage{graphicx}
\usepackage{color}
\usepackage{array}
\usepackage{tikz,pgf,pgf-pie}
\usetikzlibrary{arrows,positioning}
\usepackage{xspace}

%\usepackage{a4wide}

\definecolor{violet}{rgb}{0.5, 0, 0.8}
\definecolor{green}{rgb}{0.2 ,0.5, 0.7}
\definecolor{blue}{rgb}{0.3, 0 ,0.6}

\usepackage{multibib}

\bibliographystyle{IEEETran}
\newcites{journals,conferences,chapters,workshops,software,patents,others}{%
        Revues à comité de lecture, %
	Actes de colloques à comité de lecture, %
	Chapitre d'ouvrage,
        S\'eminaires,Logiciels,
	Brevets,Références}

\newcommand{\latex}{\LaTeX\xspace}
\newcommand{\explanes}{\textsf{expLanes}\xspace}

\usepackage[pdftex,
            linktocpage,%
            bookmarks=true,%
            colorlinks=true,%
            linkcolor=violet,%
            citecolor=green,%
            urlcolor=blue,%
            ]{hyperref}


\title{}

\author{Mathieu Lagrange}

\begin{document}

%\maketitle

\part{Notice individuelle et thèmes de recherche}

\section{Notice individuelle (10)}

\chapter{\nmu Notice \nmu de lecture} \label{chap:notice}

Ce document présente de manière synthétique mes contributions, résultats de plus de dix huit années de recherche en modélisation du signal sonore. Il est composé de trois chapitres qui peuvent être lus de manière relativement indépendante. Les lecteurs non familiers avec le traitement du signal audio numérique pourront bénéficier de quelques notions fondamentales exposées dans le chapitre dédié à la \lnameref{chap:modeles} pour mieux s'approprier le propos du chapitre dédié à la présentation de mon \lnameref{chap:themes}.

Le premier chapitre s'attache à présenter de manière synthétique mon \lnameref{chap:themes} de l'analyse computationnelle de scènes sonores à leur synthèse pour l'expérimentation en écoute artificielle et l'étude de l'impact sur la perception humaine de l'exposition à ce type de stimuli.

Le second chapitre est dédié à une présentation détaillée de la \lnameref{chap:modeles}. J'y dresse un panorama de l'évolution de l'effort de recherche déployé par la communauté dans ce domaine durant ces dernières décennies, en détaillant certaines contributions que j'ai pu y apporter et en mettant l'accent sur l'identification des verrous majeurs encore existants.

Le troisième et dernier chapitre évoque plus généralement une critique de \lnameref{chap:methode}. J'y présente le fruit de mes investigations dans la formalisation et la conception d'outils de conception de protocoles expérimentaux destinés à faciliter la recherche reproductible dans le domaine des sciences des données, ainsi que quelques réflexions concernant la dernière étape de la méthode scientifique : \lnameref{sec:pairs}.

\marginnote{Pour des raisons de lisibilité, je ferais référence à la littérature dans le corps du texte en évoquant le nom du premier auteur de la publication évoquée. La liste des auteurs ainsi que les informations bibliographiques seront systématiquement disponibles à droite, dans cette colonne.}

Je vous souhaite une bonne lecture.

  %La première est une présentation synthétique de mon parcours académique, de ma thèse de doctorat débutée en 2001 à France Télécom R\&D à mon intégration en tant que chargé de recherche CNRS au laboratoire des sciences du numérique de Nantes (LS$2$N UMR $6004$).


\section{Thèmes de recherche (20)}

Aujourd'hui incontournable dans de nombreux domaines, la simulation numérique utilisée en tant que "réplicateur rationnel" de certaines caractéristiques de l'être humain est à mon sens un fantastique outil de compréhension de l'humain en ce sens qu'il permet de mieux se confronter aux limites de notre capacité de modélisation de nos propres comportements. Cet outil ne modifie en rien les règles séculaires du questionnement scientifique rythmé par des allers et retours nécessaires entre processus inductifs et déductifs. L'outil informatique permet simplement d'accélérer considérablement la cadence. Cette puissance n'est à mon sens pas sans générer actuellement un certain aveuglement en mettant fortement l'accent sur les avancées technologiques possibles au détriment de leur inclusion nécessaire dans un questionnnement scientifique qui résistera au temps.

Je présenterai ici un état des lieux de mes travaux organisé de manière a mettre en lumière l'évolution de mon point de vue sur la recherche, de la phase d'exploration à la phase de proposition en passant par une phase plus critique, nécessaire à la définition d'orientations qui soient intimement motivés et non le résultat d'une inclusion nécesaire dans une série de thématiques à la mode.

\section{Analyse computationelle de scènes auditives (5)}

Mieux comprendre comment une

thèse, vic, houle

\subsection{"Vanilla CASA" : l'approche système expert}

\subsection{"Normalized cuts"}

\subsection{ALC}

parler de la production de similarite Bof / scarce events comme motivation



\section{Constat critique}

Au dela des avantages et inconvénients des approches disctées precedeement, des phénomènes recurrents ont pu erre observés dans les comunautés auquelles j'ai contribuer (Mir) et plus largmeent dans ce que l'on appelle auhjourd'hui "les scencies des données."

horse

Ma conclusion de ces années de tatonnement et d'exploration des différentes approches algoritmiques pour résoudre le problème posé m'a amener a faire le constat suivant :
\begin{enumerate}
  \item l'absence de formalisme expérimental (citer le kmeans en image)
  \item l'ajustement aux données, que ce soit par des approches de types série de peignes, ou de méta paramétrisation amène le plus souvent à la production de données quantitatives coincidentales et des conclusions qualitatives dont les bases expérimentales sont fragiles.

\end{enumerate}

 Il est necessaire de pouvoir comparer simplement et efficacement de nombreuses approches différentes dans un même formalisme expérimental


Je tiens à préciser que même si les challenges tels qu'ils sont pratiqués aujourd'hui sont une certes une avancées par rapport à l'approche 'mon problème, ma base de données, mon algorithme, ma métrique' ne sont en aucun cas satisfaisant pour une démarche expérimentale rigoureuse visant à répondre à une problématique scientifique. La principale raison étant la domination de l'approche "ingénieure" qui vise à répondre à une application pratique et non d'améliorer les connaissances dans un domaine scientifique donné.

\section{Méthodologie expérimentale en traitement du signal audionumérique (5)}

y=f'(x)

recherche reproductible, explanes 5

etude de f' par la construction d'un x'

definition de y

\section{La synthèse au service de l'écoute artificielle (5)}

Dcase

\section{La synthèse au service de la psychologie expérimentale (5)}

Mcgill, these de Grégoire


\section{Notice bibliographique (2004-2018)}

\chapter{ \nmu Notice bibliographique} \label{chap:biblio}

\begin{center}
Les publications référencées ici sont disponible au format pdf à cette adresse : \\ \url{https://mathieulagrange.github.io}
\end{center}

\nocitejournals{*}
%\nocitejournals{lagrangeTaslp08, Lagrange12a, LagrangeTasslp10, lagrangeTaslp06,  lagrangeJaes05, lagrangePrl09, lagrangeMta09, lagrangeJaes07,Murphy11a,Lagrange12c}

\bibliographystylejournals{plainyr-revnonnum}
\bibliographyjournals{biblio/strings,biblio/journals}

\setcounter{enumiv}{0}

\nociteconferences{*}

\bibliographystyleconferences{plainyr-revnonnum}
\bibliographyconferences{biblio/strings,biblio/conferences}

\setcounter{enumiv}{0}

\nocitechapters{*}

\bibliographystylechapters{plainyr-revnonnum}
\bibliographychapters{biblio/strings,biblio/chapters}

\setcounter{enumiv}{0}

\nociteworkshops{*}

\bibliographystyleworkshops{plainyr-revnonnum}
\bibliographyworkshops{biblio/strings,biblio/workshops}

\setcounter{enumiv}{0}

\nocitesoftware{*}

\bibliographystylesoftware{plainyr-revnonnum}
\bibliographysoftware{biblio/software}

\setcounter{enumiv}{0}

\nocitepatents{*}

\bibliographystylepatents{plainyr-revnonnum}
\bibliographypatents{biblio/patents}


\part{Modélisation long terme de signaux sonores : un état des lieux (35)}

\section{Approches sinusoïdales (5)}

\subsection{Modélisation court terme}


\section{Approches modales (5)}

\section{Approches temps fréquence modulations (10)}

\section{Approches temporelles (10)}




\end{document}
