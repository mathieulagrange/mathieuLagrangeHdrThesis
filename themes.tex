\chapter{Parcours thématique}

Je m'intéresse à l'étude de l'être humain et en particulier à ses modes de perceptions de l'environnement. Je considère la modalité sonore parce qu'elle comporte intrinsèquement un questionnement sur la dimension temporelle. Le temps est une notion complexe\marginnote{Le temps est en effet une notion abstraite qui est pour de nombreuses raisons sujet à débat dans sa définition physique même et sa perception plus encore, voir etienne klein}, je choisi donc plus précisément de me focaliser sur la notion de causalité : "Je dispose d'un passé, disponible sous forme de mémoires, qui me permet de donner un sens à ce que je perçois".

Ce sujet d'étude est intrinsèquement multi disciplinaire et investi notamment les disciplines suivantes :
\begin{enumerate}
  \item neurosciences : étude du système nerveux
  \item psycho perception : étude des phénomènes psychiques liés aux sensations et aux perceptions, généralement en lien avec le temps.
  \item sciences des données : apprentissage, traitement du signal
\end{enumerate}

Quand on présente  ce type de thématique, il est bien entendu que les deux première disciplines sont  celles à lesquelles on pense en premier. Il pourrait donc faire sens, à l'instar de deux estimés collègues, Alain de Cheveigné (directeur de recherche Cnrs au laboratoire d'audition de l'\'Ecole Normale Supérieure) et Jean-Julien Aucouturier (chargé de recherche Cnrs à l'Ircam) qui, disposant comme d'une expertise reconnue dans des thématiques relevant des sciences des données contribuent maintenant directement à l'avancée de ces deux thématiques en apportant leur expertise mais également en construisant un savoir et un savoir faire original à l'interface entre ces thématiques.

Malgré cet intérêt partagé pour les facteur humains, j'ai fait le choix de centrer mon effort de recherche en sciences des données, pour les raisons suivantes. Aujourd'hui incontournable dans de nombreux domaines, la simulation numérique utilisée en tant que "réplicateur reproductible" de certaines caractéristiques de l'être humain est à mon sens un fantastique outil de compréhension de notre humanité en ce sens qu'il nous permet de nous confronter aux limites de notre capacité de modélisation de nos propres comportements. Cet outil ne modifie en rien les règles séculaires du questionnement scientifique rythmé par des allers et retours successifs entre processus inductifs (découverte, avancées techniques, ...) et déductifs (formalisation, théorisation, ...).

L'outil informatique permet néanmoins d'accélérer considérablement la cadence. Cette accélération n'est à mon sens pas sans générer actuellement une certaine perte de méthode. En effet, en mettant trop fortement l'accent sur les avancées technologiques possibles au détriment de leur inclusion nécessaire dans un questionnement scientifique qui résistera au temps et permettra une meilleure utilisation du potentiel de ces avancées, il y a à mon sens un risque majeur de réduire notre niveau de contrôle, pourtant indispensable à la vertu. Je reste néanmoins convaincu que la modélisation numérique sera un levier important dans le défi de la connaissance de soi\marginnote{"Connais toi toi même",  Socrate y voyait plus exactement une exhortation à «prendre conscience de sa propre mesure sans tenter de rivaliser avec les dieux ».}

Candidement armé de ce projet et de ces bonnes intentions, j'ai poursuivi ces vingt dernières années un effort de recherche au sein de plusieurs institutions de recherche et d'une communauté quasi émergente au début de ma carrière à savoir le traitement du signal audio "non speech", son musical d'abord puis son environnemental. Cette communauté a d'ailleurs, par bien des aspects, suivit des étapes de maturation similaires à celles que je décrit ici.

Cette évolution a été pour moi un passage de la phase d'exploration à la phase de proposition en passant par une phase plus critique qui s'est imposée à moi comme nécessaire à la définition d'orientations qui soient intimement motivés par un questionnement et non le résultat d'une affection plus ou moins assumée avec une série de thématiques séduisantes. L'équilibre entre isolement et inclusion thématique devenant alors un exercice difficile mais néanmoins indispensable pour être à même de maximiser l'impact de mon travail dans la communauté sans en dénaturer les motivations fondatrices.

Je présenterai donc tout d'abord un état des lieux de mes travaux organisé de manière à mettre en lumière l'évolution de mon point de vue sur la recherche scientifique en modélisation numérique en général et en traitement du signal sonore en particulier. Suit une présentation plus formelle d'un panorama des modèles de signaux applicables aux données audio numériques.

% La présentation de ces thèmes ne suit pas un formalisme académique et les opinions exprimées n'engagent que moi.

\section{Analyse computationnelle de scènes auditives}

Le système auditif humain (SAH) reste en grande partie un mystère, même si son organisation physiologique est dans ses grandes lignes connue. Je négligerai volontairement ici les aspects binauraux en considérant le système auditif humain comme mono capteur, ces aspects étant des indices finalement assez faible dans notre formidable capacité à inférer une représentation interne plausible de notre environnement. Pour asseoir notre argumentaire, on supposera que le système auditif humain se décompose en quatre étapes de traitement successives :
\begin{enumerate}
  \item transfert mécanique (mono directionnel) : tympan, osselets
  \item conversion mécanique / électrique : cochlée
  \item transfert électrique (bi-directionnel) : éléments spécifique du cerveau moyen
  \item traitement : cortex auditif
\end{enumerate}

En plus de cette conversion d'une énergie mécanique vers une énergie électrique ou encore une information analogique à une information digitale, la cochlée opère une décomposition fréquentielle qui nous permet d'aisément distinguer un son grave d'un son aigu. Le fait que cette décomposition se fasse aussi tôt dans la chaîne de traitement nous indique l'importance de cette décomposition. En prenant un parti pris évolutionnaire, on peut supposer que cette distinction a eu un impact déterminant pour la survie. En effet, une large caisse de résonance aura tendance à produire un son grave si elle est mise en vibration. Être alerté rapidement de cela peut permettre d'avoir un avantage certain pour assurer sa propre survie.

Un autre élément d'importance est que la décomposition se fait sur axe logarithmique en fréquence et le signal d'amplitude est logarithmique. L'utilité du fait que l'amplitude d'un son soit perçue de manière logarithmique est relativement bien comprise, ce qui n'est pas le cas de la raison d'être de l'axe fréquentiel. Des arguments d'ordre mathématique seront donnés à ce sujet dans la section \ref{}.

% fig Gray928
% By Henry Vandyke Carter - Henry Gray (1918) Anatomy of the Human Body (See "Book" section below)Bartleby.com: Gray's Anatomy, Plate 928, Public Domain, https://commons.wikimedia.org/w/index.php?curid=566872

Il est important de noter la communication entre le cortex auditif et la cochlée est bi directionnelle. De l'information est transmise de la cochlée vers le cortex auditif (direction montante) et du cortex vers la cochlée (direction descendante). Cette direction montante est relativement aisément étudiable dans le cadre de la psycho perception. On fait écouter un stimuli à un sujet et on le questionne de manière plus ou moins explicite. La direction descendante est beaucoup plus difficile à étudier car elle implique de conditionner le sujet et de mesurer l'impact de ce conditionnement sur l'organe de réception lui même. Cela implique nécessairement une approche "neuroscience" beaucoup plus invasive et complexe à mettre en oeuvre \marginnote{Une telle expérience à pu être mise en place avec des êtres humains, montrant l'importance de ces connexions descendantes et leurs capacités à conditionner le système auditif \cite{mesgarani2012selective}.}.

A la fin du siècle dernier, les outils à disposition étant moins avancés qu'aujourd'hui, on questionnait essentiellement la perception et la cognition humaine par des approches "holistiques" qui étudiaient nos réactions à des stimulis sans pour autant investiguer l'implantation effective dans le cerveau des mécanismes responsables de ces réactions.

Pour le traitement du son, on s'accorde pour considérer que de nombreux éléments structurants sont communément utilisés par le sah pour inférer une représentation informative de la scène sonore auquel il est soumis. Pour les besoins de l'exemple, on représentera ici la sortie de la cochlée comme un spectrogramme\marginnote{C'est, du point de vue , une approximation très grossière, mais conceptuellement suffisante pour notre présent propos. Pour une présentation des différents modèles de référence, voir \cite{}}. La tâche du sah consiste alors à associer certains éléments du plan temps/fréquence à des sources données, et ce en fonction de certains critères.  On citera, dans l'ordre d'importance, l'harmonicité, la synchronicité, la proximité en fréquence, etc. L'importance de ces critères peut être modulée dans une certaine mesure par des processus descendants comme l'attention ou encore la pratique\marginnote{Helmhotlz se disait capable de dissocier les différentes harmoniques d'une note de violon grâce à l'écoute répétée du son de ses résonateurs.}.

De nombreux autres indices ont été étudiés et de nombreux modèles de perception holistiques se basant sur les règles de la gestalt ont été proposés. En formalisant ces concepts dans une théorie unifiée, appuyée par de nombreuses expériences perceptives, Bregman a fondé l'analyse de scènes auditives (ASA) \cite{bregman1994auditory}. Ces travaux ont suscités un élan d'intérêt dans la communauté traitement du signal et plusieurs modèles computationels ont été proposés comme celui de Dan Ellis\cite{ellis}, pionnier dans ce domaine communément appelé CASA, pour "Computational Auditory Scene Analysis".

En considérant le spectrogramme comme notre représentation de départ\marginnote{On néglige ici la notion de phase, comme cela est effectué dans la plupart des méthodes de séparation de sources.} en simplifiant un peu les choses, on peut considérer que le problème CASA consiste à attribuer une couleur à chacun des paniers temps/fréquence, couleur correspondant à chacune des sources d'intérêts, chaque coloriage formant un masque.

La figure \ref{}, montre le cas du masque binaire idéal\cite{}. Les zones vertes correspondent aux zones du plan temps/fréquence où la. Ces masques sont idéaux parce qu'ils sont calculés grâce à la comparaison avec les spectrogrammes des sources avant mélange. Le masque binaire idéal est donc communément utilisé comme référence dite "oracle", car elle constitue une borne supérieure.

Séduit par l'aspect neuro-inspiré et la grande diversité algorithmique des approches CASA, j'ai  poursuivi pendant quelques années cet effort de recherche en proposant plusieurs méthodes basés sur des algorithmes de regroupement d'objets, "clustering" en Anglais.

ensemble d'objets

critère

un grand nombre d'approches algorithmiques sont disponible pour résoudre ce problème, le plus connu pour les problèmes de grande taille en nombre d'objets et d'un nombre de classe petit devant le nombre d'objet etant le kmeans classique dont l'efficacite reside dans le fait que la fonction de cout se base sur un produit scalaire.

Cet algorithme doit sa popularité à son efficace, mais s'avère inopérant dans notre cas, les relations casa n'étant pas exprimables sous forme de produit scalaire.

l'algorithme des normalized cuts, popularisé en traitement d'image \cite{shi-malik}, permet de résoudre le problème du regroupement d'objets dont seules les relations entretenues entre ces objets sont connues, plus précisément que leur dissimilarité soit non pas une distance mais un noyau.

Muni d'un tel outil, il reste à spécifier les relations entretenues par les objets considérés. Pour la segmentation d'image, la relation utilisée dans consiste en une couleur avec contrainte de localité :
\begin{equation}
t
\end{equation}
Dans une première étude menée à l'Université de Victoria en collaboration avec George Tzanetakis, j'ai considéré le problème suivant. Soit une version compacte du spectrogramme (un modèle sinusoidal à court terme, voir la Section \ref{} qui lui est dédiée.), comment définir mathématiquement quelques relations ASA et les combiner entre elles, avec pour objectif d'isoler la partie mélodique d'un enregistrement musical polyphonique. Dans le cas d'une mélodie, la relation la plus importante consiste en la relation harmonique qui lie les différents composantes de la source mélodique, voir Figure \ref{}.

Cette première approche a reçu un bon accueil de la communauté\cite{conf, lagrangeTaslp08}, et son implantation a été diffusée dans le cadre computationel Marsyas \cite{}. Dans le cadre d'une ANR JCJC Houle \url{}, j'ai, au sein de l'Ircam, et en collaboration avec Mathias Rossignol, tenté d'améliorer cette approche en prenant le postulat suivant : toute segmentation est effectuée sur la base de contraintes de localité dans deux espaces disjoints, une espace analogue à l'espace/temps, et un autre relatif à notre forme. Ces deux types d'espaces induisent respectivement des relations de proximité et de similarité. Prenons l'exemple d'une cuillère à café placée dans une tasse et une cuillère à soupe placée plus loin sur une table. La cuillère à café étant dans la tasse, elle est proche spatialement de la tasse, et elle est également similaire à la cuillère à soupe située un peu plus loin.

Dans le cas de la segmentation d'une image, Shi et al ont implanté la notion de similarité par une différence de colorimétrie entre les deux pixels comparés. La notion de proximité est implanté par un seuil, voir Equation \ref{eq-shi}.

En partant d'objets élémentaire comme le pixel d'une image ou le panier temps fréquence d'un spectrogramme, il est évident que la notion de proximité doit prédominer, car la notion de similarité se base sur une quantité d'information assez faible.  En revanche, plus on considère des objets complexes, plus l'importance de la similarité devient grande devant la notion de localité. Fort de ce constat, nous avons donc chercher à concevoir un algorithme de structuration hiérarchique basé sur l'expression de ces deux critères.

Mais avant cela, nous devions disposer d'un algorithme de clustering alternatif aux normalized cuts. Cet algorithme est en effet couteux, principalement à cause de l'étape de décomposition en valeurs singulières ce qui contraint l'horizon temporel exploitable. Il nécessite également un noyau pour garantir sa convergence, ce qui n'est pas le cas en général.

Brian Kullis montre en  l'objectif des coupures normalisées peut êtee formuler comme une fonction de cout optimisable par un algorithme à noyau, ce qui permet d'approcher le problème du clustering avec une complexité considérablement réduite. Il reste que la convergence de ce type d'approche n'est vérifiée que dans le cas où la matrice de dissimilarité est semi définie positive. Les relations CASA étant complexes, il nous a paru important de lever cette contrainte\marginnote{Il est possible de manipuler la matrice pour la rendre semi définie positive \cite{} mais l'impact de cette manipulation sur le regroupement obtenu est difficile à contrôler.}.

Nous avons donc développé un algorithme de clustering dont la convergence est assuré pour tout type de matrice de dissimilarité, pour peut qu'elle soit symétrique. Cette approche, même si elle considère des relations bien connues, n'avait, à notre connaissance, pas été formalisée complètement en un algorithme effectif. Son évaluation sur un large ensemble de corpus \cite{keogh} montre des performances équivalentes aux méthodes à noyau, avec un coût de calcul et une empreinte mémoire plus faible, ce qui la rend intéressante pour des problèmes de grande taille\cite{rossignolKaverages}.

A partir de cet algorithme adapté à nos besoins, nous avons proposé un algorithmique de segmentation de scènes sonores basé sur le principe de regroupement similarité / proximité\cite{rossignolhal-01122006}.

% alc_one_level.png

Schéma décrivant les principaux traitements utilisés pour passer du niveau $i$ au suivant. Une étape de classification non supervisée des fragments $o_i$ produit des labels $c_i$ qui sont utilisés comme information structurelle, qui combiné avec un critère de continuité spectrale, guide la détermination des fragments $o_{i+1}$ du niveau suivant.

\begin{enumerate}
\item \textbf{Initialisation} par une première classification : regrouper les trames successives qui ont une grande similarité spectrale pour déterminer les fragments de niveau 0
\item \textbf{Répéter} pour le nombre spécifié de niveaux :
  \begin{itemize}
  \item Calcul des similarités entre fragments de niveau $i$
  \item Classifier ces fragments en classes $C_i$ en fonction de ces similarités
  \item Calcul de l'information mutuelle entre les $C_i$, en se basant sur le degré de séquentialité entre $C_i$ et $C_j$ :
    \[ MI \left( C_i, C_j \right) = \log \left(\frac{p\left(C_iC_j\right)}{p\left(C_i\right)p\left(C_j\right)}\right) \]
    où $p(C)$ désigne la probabilité qu'un fragment appartienne à une classe donnée, et $p(C_iC_j)$ est la probabilité que deux éléments soient consécutifs.
  \item Générer une courbe de décision le long de l'axe temporel fonction de:
    \begin{itemize}
    \item l'information mutuelle entre les classes des deux fragments consécutifs,
    \item la continuité spectrale, définie comme la similarité spectrale entre la fin d'un fragement et le début du fragment suivant.
      \end{itemize}
  \item Segmenter la séquence de fragments en fonction de la courbe de décision, en combinant les fragments situés entre les minima locaux de cette courbe de décision.
  \end{itemize}
\end{enumerate}

%alcéchantillon
\label{fig:alcéchantillon}

Exemple de segmentation obtenue. La forme d'onde est donnée pour référence avec un code couleur. A chaque niveau, la largeur des blocs indique le support temporel du fragment associé, la hauteur indique la valeur de la fonction de décision à ce niveau, et la couleur le label associé à la classe à laquelle le fragment appartient.

% smoke.png
Segmentation du morceau de Deep Purple \emph{Smoke on the Water} (\emph{Machine Head}, Purple Records / EMI, 1972).

La Figure~\ref{fig:smoke} présente la segmentation de l'introduction du célèbre morceau de
Deep Purple \emph{Smoke on the Water}. La même phrase musicale est répétée deux fois avec quelques variations d'interprétation. Des motifs émergent à partir du niveau 4 où les motifs élémentaire du riff sont révélés.

Ces années d'exploration des approches casa m'a permis de parcourir de larges champs d'expertise scientifiques, de l'informatique au traitement du signal, de la psycho-percpetion aux neurosciences, mais cela a surtout été pour moi l'occasion de mieux saisir certaines problématiques fondamentales en sciences des données.
\begin{enumerate}
  \item Comment dévoiler la structure intrinsèque aux données à partir d'observations bruitées ?
  \item Comment exploiter l'approche compositionnelle de notre compréhension du monde ?
  \item Quel rôle peut jouer la notion de mémoire dans les processus pré cités ?
\end{enumerate}
Ces questions continuent de nourrir mes recherches car les solutions auxquelles j'ai contribué ne sont, à mon sens, en aucun cas satisfaisantes. La traduction algorithmique de contraintes de structuration est un art difficile, car elle nécessite un effort important pour fiabiliser les algorithmes. La gestion de l'interaction entre les différentes contraintes de structuration l'est aussi. Enfin, la notion de mémoire est ici exploitée de manière implicite en considérant la redondance d'information dans la scène observée. Avec les horizons temporels considérés à l'époque pour rendre l'approche tractable, cette approche s'est révélé insuffisant.  %L'utilisation d'un schéma alterné dans l'algorithme alc a permit de mieux contrôler cette interaction mais cela reste au fond arbitraire.

Au delà de ces problématiques propres aux approches casa, les approches de séparation de sources se basent généralement sur un traitement en deux temps : 1) représentation du signal audio sous forme de spectrogramme, 2) segmentation. On verra dans le chapitre \ref{}, dédié aux modèles de sons les limites ce que cela impose à la performance de ces approches.

\section{La synthèse pour l'écoute artificielle}

Cette dernière étude a également été pour moi l'occasion de réfléchir au protocole d'évaluation, et en particulier aux données utilisées pour évaluer les algorithmes proposés. Il est évident que les données constituant la référence dans le domaine des sciences des données, son influence sur la structure algorithmique résultante est déterminante. Cette question est donc cruciale.

Dans le domaine du traitement du signal, on fait souvent la différence entre des signaux "réels" et des signaux "synthétiques". Les signaux réels sont issus d'un processus d'acquisition, plus ou moins maîtrisé, on parle souvent aussi de données brutes. La seule manipulation que l'on puisse alors faire pour mieux contrôler ces données, c'est de réduire la taille du corpus. Cette réduction peut se faire selon les différents axes de structuration de ce corpus\marginnote{Dans le cas d'un corpus structuré sous forme de classes par exemple, on peut choisir de supprimer une ou plusieurs classes, ou de supprimer des éléments dans chaque classes de manière à obtenir un corpus balancé, \textit{i.e.} avec un même nombre d'éléments par classe.}. Elle peut se faire également de manière globale, pour enlever des données aberrantes par exemple. Dans tout les cas, il est important de considérer que ces procédures sont délicates car difficile à motiver\marginnote{Le risque du "cherry picking", qui consiste à trouver les données qui fonctionnent bien avec l'approche algorithmique défendue est toujours présent}.

\marginnote{Je n'évoquerai pas ici la question importante de la représentativité du corpus et de la division du corpus complet en corpus d'entrainement, de test, et de validation. Le lecteur peut se référer à l'excellent cours de Stéphane Mallat sur la notion de risque en apprentissage pour de plus amples détails sur le sujet.}.


Les signaux synthétiques sont, eux, totalement contrôlés. Les auteurs proposent un modèle de signal, et en utilisant ce modèle de signal pour générer des exemples, on montre que la méthode d'estimation proposée conjointement à ce modèle est effective sur ces exemples. Cela permet de faire une démonstration de faisabilité du modèle, mais n'est pas suffisante pour démontrer l'applicabilité de ce modèle à des problématiques concrètes.

La méthode couramment prise est donc d'estimer ce niveau d'applicabilité
en considérant les performances obtenues par le système en considérant des données brutes. Le principal risque de cette foi aveugle dans les résultats obtenus en utilisant les données brutes, est précisément que nous disposons souvent d'un contrôle très faible sur ces données, et ce qui à permis de les obtenir, \textit{i.e.} le processus d'aquisition, le processus d'annotation, etc.

Prenons l'exemple du calcul d'une statistique telle que la moyenne, un ensemble d'outil nous permet de valider la pertinence de cette statistique, distribution, écart à la médiane, quartiles, etc.

Pour un modèle avec plusieurs millions de paramètres libres comme c'est le cas pour les architecrtures de traitement de l'information telles que les réseaux de neurones profonds, les outils d'inspection sont nettement plus ardu à mettre en place. La valeur de la métrique de qualité obtenue par le système doit donc être appréciée avec beaucoup plus de mesure.

Le modèle proposé est'il pertinent ? Est ce que l'information captée par la méthode d'estimation se rapporte à des phénomènes d'intérêts ? Pour tout problème qui sort de la trivialité, nous ne disposons pas à l'heure actuelle d'outils qui permettent de répondre à ces questions.

La revue par les pairs\marginnote{Dans une forme moderne incluant la notion de recherche reproductible comme celle dévelopée dans le Chapitre \ref{}} constitue probablement la seule approche efficace à l'heure actuelle. Précisons ce propos avec deux exemples.

La communauté d'acoustique environmentale se base principalement sur l'étude de mesures acoustiques tels que le Laeq. De nombreuses études issues de cette communauté montrent néanmoins que ces mesures ne corrèlent pas toujours avec des jugements perceptifs.

Le fait que l'on puisse reconnaitre l'ambience sonore et l'affecter à une typologie de lieu a été très bien recue par la communauté et pris pour fait acquis, en partie parce que cette étude appliquait des techniques mal connues par les membres de cette communauté.

Intrigué par ces excellent résultats qui contrastaient avec les résultats obtenus dans d'autres études, j'ai entrepris de répliquer ces travaux. Cette tentative de réplication, effectuée en collaboration avec les auteurs, à été fructueuse et m'a permis de comprendre que les bons résultats obtenus étaient dus au protocole expérimental\cite{lagrange2015}. Les scènes présentes dans le corpus ont été enregistrées durant une période longue (environ une heure) et découpées en tronçons d'une minute. Etant donné la stationnarité de ce type de scènes sonores en terme de composition, deux tronçons successifs sont la plupart du temps quasiment identiques en terme de contenu. Le protocole expérimental n'empêchant pas que l'algorithme d'apprentissage exploite des tronçons d'une scène donnée pour prédire le label d'un tronçon de cette même scène, le problème s'en trouvait, dans son incarnation numérique, considérablement simplifié. \marginnote{Ce phénomène est similaire à l'effet "album" en musique. Les conditions d'enregistrements (microphones, mixage, mastering) sont souvent spécifique à l'album enregistré, spécificité que les algorithmes d'apprentissage exploitent pour minimiser leur erreur de prédiction. Pour minimiser cet effet, il convient alors de ne pas permettre d'avoir des chansons d'un même album dans les corpus d'apprentissage et de test.}.

Le scattering d'ondelettes, modèle de signal particulèrement interessant qui sera evoqué dans la Section \ref{} a été utilisé dans le cadre de la classification de genre musical. Le scattering est en charge de produire une représentation du signal sonore qui soit informative mais de faible dimensionalité. Cette représentation est ensuite utilisé par algorithme de classification supervisée prédire quel genre parmi une typologie déterminé, est associé à un morceau de musique ou une partie de ce morceau.

Dans une étude de réplication et d'analyse de la robustesse de système de prédiction du genre musical\cite{rodriguez2016analysing}, Rodriguez et Sturm montrent que les bonnes performances des méthodes de l'état de l'art se basent (au moins en partie) sur 1) une base de données mal structurée, et 2) la disponibilité d'informations inaudible pour l'être humain.

La base de données GTZAN proposée par George Tzanetakis a longtemps été une référence sur le sujet, car elle a été conçue pour évaluer le premier système de prédiction du genre musical\cite{tzanetakis2002musical}. Plusieurs études ont montrées les limites de cette base de données en terme de structure. L'utilisation d'une version améliorée de la base fait décroitre les performances des méthodes testées de plus de 20 \%.

Egalement, les algorithmes d'apprentissage exploitent des informations présentes dans les bandes de fréquences inférieures à 20 Hz. Sur une version où les bandes fréquences de fréquence inférieures à 20Hz sont fortement atténuées, les performances se réduisent de 15 \% environ\marginnote{L'oreille ne percevant pas les signaux en dessous de 50 Hz, on peut dire que le système d'apprentissage se focalise sur des composantes inaudibles pour l'être humain}.

Ces deux exemples montrent combien il est difficile d'apprécier la performance d'un algorithme de traitement de données. Dans une posture particulièrement sceptique, Bob Sturm désigne tout ces algorithmes comme des "horses" potentiels \marginnote{Cette métaphore fait référence à "Clever Hans", un cheval supposément capable de compter. Une personne lui donnait une opération mathématique à résoudre, et Hans tapait du sabot au sol et s'arrêtait de taper quand le compte de coups de sabot correspondait à la solution de l'opération. Confronté par un protocole rigoureux, il s'est avéré que Hans n'était capable de donner la bonne réponse que si la personne qui lui posait la question connaissait la réponse. En ce sens, Hans faisait preuve d'une intelligence sensible extrêmement fine pour pouvoir détecter les subtiles modifications de posture à l'approche du bon nombre de coups de sabots qui indiquait à Hans l'approche de sa récompense, mais cette intelligence n'était assurément pas de type mathématique.}.

Néanmoins, le fait d'utiliser une corrélation fortuite entre différentes informations présentes dans les données, n'est pas forcément dommageable\cite{lagrangehal-01635373}. Il est également extrêmement courant d'optimiser une fonction de coût donnée que l'on estime\marginnote{Pour des raisons axiomatiques concernant le type des données, ou empiriques} corrélée avec l'objectif principal. Dans la mesure. Il est par contre fondamental d'avoir conscience de ces problématiques pour être à même d'apprécier avec un regard suffisamment critique les mesures de performance dérivées de tel ou tel algorithme ou protocole expérimental.

On a donc vu que l'usage de données synthétiques ne permettait qu'une validation du concept, et que l'usage de données brutes doivent être encadrées avec soin et que, même si le but d'un algorithme de traitement de données est d'être capable \textit{in fine} de traiter des données brutes, le niveau de contrôle est extrêmement faible, ce qui réduit drastiquement la confiance que l'on peut avoir des métriques de performance obtenues\marginnote{Je ne détaillerai pas ici le problème de l'annotation des données brutes qui se pose également, et qui, dans le cas de certaines tâches peut s'avérer particulièrement problématique. Deux exemples permettront de nous fixer les idées. Comment spécifier le genre d'un morceau de musique comme "Psyché Rock" de Pierre Henry ? Comment annoter la fin d'un évènement sonore dans une scène sonore complexe ?}.

Dans le but de mitiger ces problèmes, il semble convenable de complémenter ces deux approches par une approche intermédiaire en considérant ce que j'appellerai des données "simulées". On verra dans la suite que, même si ces données simulées ne constituent pas une validation en soi, elles sont un outil particulièrement efficace pour une approche méthodologique mieux contrôlée, car fondée sur un questionnement scientifique.

Prenons le cas de la tâche de détection d'évènements sonores dans des scènes sonores complexes. Ces scènes sonores sont composées d'évènements sonores qui peuvent être présent de manière simultanés ou non, un bruit de fond plus ou moins élevé est présent. On souhaite que le système soit capable :
\begin{enumerate}
  \item d'être robuste à la présence du bruit de fond, \textit{i.e.} ne pas détecter d'évènements quand il n'y a que du bruit de fond dans l'intervalle d'observation et ne pas se tromper de classes d'évènements lorsque le bruit de fond se superpose avec l'évènement  dans l'intervalle d'observation;
  \item d'être capable de gérer la diversité intra classe de chaque classe d'évènements sonores et de type de bruit de fond,  \textit{i.e.} dans le cas de la classe "frappe sur clavier d'ordinateur", le système doit être capable de gérer une grande diversité de marques de claviers et de styles de frappes;
  \item être robuste aux changements de conditions d'enregistrements,  \textit{i.e.} le type de microphone, la qualité de l'enregistrement et les propriétés de réverbération de la pièce doivent pouvoir changer sans influer sur les performances du système.
\end{enumerate}

Evaluer ces éléments nécessite d'avoir a disposition des corpus séparés avec dans le premier cas, un niveau de bruit de fond qui augmente, dans le deuxième cas, une diversité qui augmente, et dans le troisième cas des conditions d'enregistrements qui changent. Bien entendu, la prise de son et l'annotation de scènes sonores brutes pour chacune de ces conditions expérimentales est possible mais particulièrement difficile à mettre en place pour obtenir le niveau de précision souhaité.

Nous avons considéré un modèle de scènes sonores qui permet d'aisément contrôler ces paramètres\marginnote{Une implantation de ce modèle dédiée à l'évaluation a été réalisée en Matlab \url{}. Un outil aux fonctionnalités proches a été développé en Python par Justin Salamon \url{}.}. Le modèle proposé se présente comme suit. Pour une scène $s$ composée de $z$ classes de sons, on a :

\begin{equation}
s(n)=\sum_{i=1}^{z}p_i(n)
\end{equation}

avec $n$ un indice temporel discret, et $p_i$ la piste correspondant à la classe $c_i$. La classe $c_i$ est composée de $\vert c_i\vert$ échantillons $c_{i,m}$, $1<m<\vert c_i\vert$.

Soit $\mathcal{U}(x,y)$, une distribution uniforme d'entiers allant de $x$ à $y$, avec $x<y$. On définit $E_j^i$ ($j=\lbrace 1,2,\ldots,k_i\rbrace$) une suite de variables aléatoires indépendantes et identiquement distribuées (iid) suivant la loi $\mathcal{U}(1,\vert c_i \vert)$.

\begin{equation}
E_j^i \textrm{ iid : } \mathcal{U}(1,\vert c_i \vert) \quad \forall j
\end{equation}

Une piste $p_i$ est vue comme une séquence de $k_i$ échantillons d'événements. On définit $e_j^i(n)$, un événement choisi aléatoirement parmi les $\vert c_i\vert$ échantillons de la classe $c_i$ :

\begin{equation}
e_j^i=c_{i,E_j^i}
\end{equation}

On définit $A^i_j$ et $T^i_j$ ($j=\lbrace 1,2,\ldots,k_i\rbrace$), deux suites de variables aléatoires iid. Pour chaque piste $p_i$, les $A^i_j$ sont les facteurs d'amplitudes appliqués aux événements $e_j^i$, dont nous modélisons la distribution, par souci de simplicité, par une loi normale de moyenne $\mu_a^i$ et de variance $\sigma_a^i$. De même, pour chaque piste $p_i$, les $T_j^i$ sont les espacements inter-\emph{onsets}, lesquels suivent une loi normale de moyenne $\mu_t^i$ et de variance $\sigma_t^i$.

Soit $\mathcal{N}(\mu,\sigma)$, une distribution normale de moyenne $\mu$ et de variance $\sigma$, on a alors :

\begin{equation}
\label{eq:ch4_eq1}
A_j^i \textrm{ iid : } \mathcal{N}(\mu_a^{i},\sigma_a^{i}) \quad \forall j \quad \textrm{ et } \quad T_j^i \textrm{ iid : } \mathcal{N}({\mu_t^{i},\sigma_t^{i}}) \quad \forall j
\end{equation}

Lors de la génération d'une scène, les valeurs des $A^i_j$ et $T_j^i$ sont déterminées par tirage aléatoire, suivant les distributions correspondantes.

On définit $u^i$ et $v^i$ les indices temporels de début et de fin de chaque piste $p_i$ respectivement.

Une piste $p_i$ se définit alors comme suit:

\begin{equation}
\label{eq:ch4_eq2}
p_{i}(n)= \sum_{j=1}^{k_i} A_j^i e_j^i(n-n_j^i) \quad \textrm{ avec } \quad n_j^i=n_{j-1}^i + T_j^i
\end{equation}

où, par convention, $n^i_0=u^i$ et $p_i(n)=0$ si $n>v^i$.

Les paramètres du modèle sont, $\mu_a^i$, $\sigma_a^i$, $\mu_t^i$, $\sigma_t^i$, $u^i$ et $v^i$, et doivent être fixés pour chaque piste $p_i$. La figure~\ref{fig:modelSequence} offre une illustration de l'action des paramètres introduits.

Pour les bruits de fonds, deux distinctions sont à observer avec le modèle défini précédemment:

\begin{enumerate}
\item afin d'éviter toute sensation de discontinuité, deux échantillons de texture sont concaténés en considérant un recouvrement fixé, sur lequel est appliqué un fondu enchaîné (\emph{cross-fade}) à valeur d'énergie constante entre les échantillons, afin de réduire les discontinuités;
\item il n'y a qu'un facteur d'amplitude par piste ($A^i \textrm{ : } \mathcal{N}(\mu_a^{i},\sigma_a^{i})$), sa valeur s'appliquant à tous les échantillons.
\end{enumerate}

Note sur la qualité acoustique qui nous poussera à investiguer des aspects de perception avec ref à la section suivante

\begin{figure}[t]
        \graphicspath{{figures/}}
        \def\svgwidth{\linewidth}
        \input{figures/controlParameters2.pdf_tex}
       \caption{Représentation schématisée des pistes du modèle de scènes sonores.}\label{fig:modelSequence}
\end{figure}

Nous avons mis en application ce modèle dans la première édition du challenge dcase\marginnote{Ce challenge, soutenu par la société IEEE ASSP, se propose de promouvoir l'activité scientifique et technique autour de la détection de scènes et d'évènements dans un contexte sonore environnemental.}. Dans ce cadre, une tâche de détection d'évènements sonores dédiée à l'approche par simulation a été proposée\cite{stowellhal-01253912} à partir d'un un corpus de scènes simulées grâce à l'outil simScene. Ce corpus utilisait des échantillons sonores et des bruits de fonds enregistrés à l'université de Queen Mary (qmul). Ce corpus sera nommée \emph{testQ}.

Nous avons ensuite voulu tester la capacité de généralisation de ces algorithmes, \textit{i.e.}~leur aptitude à maintenir des performances de détection similaires sur plusieurs corpus de scènes présentant des conditions expérimentales différentes. Pour ce faire, nous avons considéré un autre corpus d'échantillons sonores et de bruits de fonds  cette fois ci enregistrés à l'irccyn, Nantes. Avec la permission des auteurs des différents systèmes soumis au challenge, ces derniers sont évalués sur les corpus de scènes simulées, en utilisant les mêmes serveurs de calcul que ceux utilisés pour le challenge dcase 2013\marginnote{De nombreuses métriques sont disponibles pour évaluer ce type de système. On se concentrera ici sur la métrique $Fcw_{eb}$, une F-mesure, calculée en prenant en compte les débuts des événements, et en normalisant les résultats par classe.}.

NOTE: on generait des corpus d'apprentissage ou non ?

La capacité de généralisation est considérée suivant trois angles :
\begin{enumerate}
\item robustesse à la diversité des échantillons: évaluer la capacité de généralisation sur des corpus de scènes possédant les mêmes caractéristiques structurelles (intensité sonore des échantillons, positionnement des échantillons), mais composés d'une sélection de échantillons différents.
\item robustesse à la diversité structurelle: évaluer la capacité de généralisation sur des corpus de scènes composés des mêmes échantillons, mais dont les caractéristiques structurelles (intensité sonore des échantillons, positionnement/espacement moyen des échantillons) diffèrent;
\item robustesse au niveau du bruit de fond. Ce niveau est mesuré en termes relatifs, avec la mesure ebr (pour "event to background ratio"), exprimée en déciBels.
\end{enumerate}

Dans le premier cas, on considère une structure de scène de référence (correspondant aux annotations effectuées sur \emph{testQ} pour générer deux nouveaux corpus \emph{insQ}, et \emph{insI}. Le premier est généré à partir d'échantillons enregistrés à qmul, le second à partir d'échantillons enregistrés à l'irccyn. Le premier a une structure et un contenu sonore équivalent à \emph{testQ}, une certaine équivalence des performances est attendue. Comme on peut le voir sur la Figure \ref{} c'est globalement le cas, sauf pour deux méthodes qui se retrouvent avec des performances drastiquement réduites\marginnote{Une analyse d'erreur a montrée que ces changements de comportements étaient dûs à des phénomènes de sur-apprentissage\cite{lafayhal-01111381}.}, ce qui valide le protocole expérimental. Le second corpus (\emph{insI}) permet d'évaluer la capacité de généralisation aux types d'échantillons. On voit sur la Figure \ref{fig:irccyn} que, à l'exception de le système SCS, tout les autres ne sont pas capable de généraliser et se retrouvent avec des performances équivalentes à la référence ("Baseline").

Dans le deuxième cas, on considère le modèle de génération de scènes précédemment décrit pour générer deux corpus avec une certaine diversité structurelle. Le premier considère des échantillons enregistrés à qmul et le second à l'irccyn. Les paramètres du modèle ($\mu_a$, $\sigma_a$, $\mu_t$, $\sigma_t$, ...) sont estimés à partir des annotations du corpus \emph{testQ}. Les résultats montrent une fiable dépendance des algorithmes à la structure temporelle des scènes analysées.

Dans le troisième cas, on considère quatre corpus, avec la même structure temporelle, les mêmes échantillons sonores, mais avec un niveau relatif de bruit de fond différent. L'ebr varie de -12 dB à 6 dB, le niveau de référence (équivalent à celui mesuré sur le corpus \emph{testQ})  étant de 0 dB. Les résultats, affichés sur la Figure \ref{fig:ebr}, montrent une forte sensibilité au niveau d'ebr pour toute les systèmes, sauf pour le système SCS, doté d'un algorithme de gestion du bruit de fond particulièrement efficace.

\begin{figure}[t]
\includegraphics[width=1\columnwidth]{figures/dcase2013_1}
\caption{Performances des systèmes évalués dans le cadre du challenge DCASE 2013 sur les corpus QMUL et IRCCYN en considérant la métrique $Fcw_{eb}$.}
\label{fig:irccyn}
\end{figure}

\begin{figure}[t]
\begin{center}
\includegraphics[width=1\columnwidth]{figures/dcase2013_2}
\caption{Performances des systèmes évalués dans le cadre du challenge DCASE 2013 sur les corpus \emph{instance-QMUL} simulés avec différents $EBR$ ($6$, $0$, $-6$ et $-12dB$).}
\label{fig:ebr}
\end{center}
\end{figure}

dcase 2016 controle fin de la complexité

analyse poussée des résultats \cite{lafayhal-01635414}

Dcase  \cite{mesa} Data augmentation

Je terminerai par une réflexion issue d'un panel organisé lors du Workshop dcase 2018. Le constat était fait pour une large communauté de chercheur sur le fait les approches d'apprentissage profond et les méthodes ensemblistes\marginnote{Au sens ensemble de classifieurs forts.} apportent certes des performances élevées, mais finalement peu d'éléments de réflexions et compréhensions.

A mon avis, si la question posée est d'ordre pratique, ce n'est pas problématique que les réponses données ne soient que techniques. Encore faut t'il que ces réponses aient un intérêt pour la communauté. Pour cela, la poursuite d'une méthode rigoureuse d'investigation reste d'importance, mais elle ne doit pas être confondue avec une approche d'investigation scientifique qui doit à mon avis, elle se baser sur une question. De la question nait un corpus d'observation qui peut servir ensuite de base à l'investigation. Cette investigation peut comporter une partie technique, mais elle reste au service d'un questionnement qui lui est, par essence, scientifique. Il est donc du devoir de celui qui propose une tâche, si il y a lieu, de réfléchir à la pertinence de cette tâche au vu d'un certain questionnement pertinent pour la communauté.

\section{La synthèse pour la psychologie expérimentale}

Même si, comme je l'ai explicité dans l'introduction de ce chapitre, je situe mon centre de gravité thématique dans le domaine des sciences des données, j'ai eu le plaisir de contribuer également dans le domaine de la psychologie expérimentale avec Grégoire Lafay et Nicolas Misdariis. Nous avons souhaités questionner la notion d'agrément sonore en zone urbaine, sujet d'une grande importance sanitaire \cite{europe}.

Pour cela, le modèle de scènes décrit précédemment, de part son réalisme et sa simplicité de manipulation s'est révélé être un outil pertinent pour questionner d'une manière nouvelle des notions de psycho perception comme l'agrément. L'étude de référence sur le sujet de l'agrément sonore en zone urbaine \cite{guastavino} considère un paradigme expérimental classique fondé sur la description faite par le sujet de l'objet d'étude. Chaque sujet est invité à verbaliser les propriétés d'un environnement sonore idéal ou non idéal. Ces descriptions textuelles sont ensuite traitées par une analyse psycho linguistique, des invariants dans les descriptions données par les sujets sont dégagés qui permettent alors de conclure quand aux propriétés de ces environnements.

On note ici que ces environnements sont pensés par le sujet, puis décrit, ce qui place l'étude dans de la cadre de la théorie classique de la cognition, à savoir que les percepts subissent une étape de transduction en informations amodales que l'on peut questionner par le langage. Des théories alternatives existent, comme la théorie ancrée \cite{barsalou2010grounded} dont le propos est de ne pas distinguer perception et cognition en terme d'objets manipulés, mais en terme d'objectifs. Elle met à ce titre l'accent sur le contexte et la notion d'expérience, d'"embodiement". Dans cette théorie, les percepts sont le fondement, la matière brute des constructions cognitives qui en découlent. La cognition n'est plus en charge d'une transduction mais d'une extraction d'invariants qui serviront à construire des représentation de plus en plus abstraite. Notre proposition s'inscrit dans ce schéma de pensée, en demandant non plus au sujet de verbaliser son image mentale mais de "construire" une représentation de l'image mentale que le sujet a d'un environnement sonore idéal\marginnote{Les scènes produites par les sujets sont disponibles ici \url{}. Comme on peut s'en rendre compte à l'écoute, les scènes construites par les sujets sont plausibles mais ne sont pas pour autant réalistes au sens stricte du terme. On notera que l'on cherche ici, et ce avec un certain nombre de contraintes (durée de la scène, choix des éléments de base, paradigme de séquencement), en quelque sorte à exemplifier ces images mentales de haut niveau. Sans parler de caricatures, le côté pittoresque des scènes produites, notamment pour les scènes idéales, inscrit donc bien le protocole proposé dans la théorie ancrée de la perception.}.

Cette étude a permis de confirmer certains résultats obtenus dans l'étude de Guastavino \cite{guastavino2006ideal} et d'en questionner d'autres. On citera par exemple, la présence systématique des bus dans les scènes idéales "pensées", alors que notre étude les placent plutôt dans les scènes non idéales. On voit ici, comment d'un point de vue conceptuel, les transports en commun sont positivement connotés dans un environnement urbain, alors qu'ils restent des objets mobiles lourds avec une cylindrée conséquente, générant des niveaux de bruits mécaniques élevés\cite{lafayhal-01300399}.

Au travers de ces deux exemples, on voit comment la capacité à manipuler la matière sonore nous permet de mieux questionner notre connaissance de la manière dont nous, humains, percevons et faisons sens de notre environnement. Le modèle présenté ici est à proprement parler un modèle de séquencement de sons, donc d'assez haut niveau. Mais ce n'est pas suffisant, nous devons disposer de modèles de sons qui soient capables de manipuler finement la matière sonore, et ce avec de bonnes propriétés. Je dédie donc la partie suivante à l'explicitation de ces propriétés qui nous guideront ensuite pour dresser un panorama des modèles de sons à notre disposition.
