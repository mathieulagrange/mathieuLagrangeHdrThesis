Aujourd'hui incontournable dans de nombreux domaines, la simulation numérique utilisée en tant que "réplicateur rationnel" de certaines caractéristiques de l'être humain est à mon sens un fantastique outil de compréhension de l'humain en ce sens qu'il permet de mieux se confronter aux limites de notre capacité de modélisation de nos propres comportements. Cet outil ne modifie en rien les règles séculaires du questionnement scientifique rythmé par des allers et retours nécessaires entre processus inductifs et déductifs. L'outil informatique permet simplement d'accélérer considérablement la cadence. Cette puissance n'est à mon sens pas sans générer actuellement un certain aveuglement en mettant fortement l'accent sur les avancées technologiques possibles au détriment de leur inclusion nécessaire dans un questionnnement scientifique qui résistera au temps.

Je présenterai ici un état des lieux de mes travaux organisé de manière a mettre en lumière l'évolution de mon point de vue sur la recherche, de la phase d'exploration à la phase de proposition en passant par une phase plus critique, nécessaire à la définition d'orientations qui soient intimement motivés et non le résultat d'une inclusion nécesaire dans une série de thématiques à la mode.

\section{Analyse computationelle de scènes auditives (5)}

Mieux comprendre comment une

thèse, vic, houle

\subsection{"Vanilla CASA" : l'approche système expert}

\subsection{"Normalized cuts"}

\subsection{ALC}

parler de la production de similarite Bof / scarce events comme motivation



\section{Constat critique}

Au dela des avantages et inconvénients des approches disctées precedeement, des phénomènes recurrents ont pu erre observés dans les comunautés auquelles j'ai contribuer (Mir) et plus largmeent dans ce que l'on appelle auhjourd'hui "les scencies des données."

horse

Ma conclusion de ces années de tatonnement et d'exploration des différentes approches algoritmiques pour résoudre le problème posé m'a amener a faire le constat suivant :
\begin{enumerate}
  \item l'absence de formalisme expérimental (citer le kmeans en image)
  \item l'ajustement aux données, que ce soit par des approches de types série de peignes, ou de méta paramétrisation amène le plus souvent à la production de données quantitatives coincidentales et des conclusions qualitatives dont les bases expérimentales sont fragiles.

\end{enumerate}

 Il est necessaire de pouvoir comparer simplement et efficacement de nombreuses approches différentes dans un même formalisme expérimental


Je tiens à préciser que même si les challenges tels qu'ils sont pratiqués aujourd'hui sont une certes une avancées par rapport à l'approche 'mon problème, ma base de données, mon algorithme, ma métrique' ne sont en aucun cas satisfaisant pour une démarche expérimentale rigoureuse visant à répondre à une problématique scientifique. La principale raison étant la domination de l'approche "ingénieure" qui vise à répondre à une application pratique et non d'améliorer les connaissances dans un domaine scientifique donné.

\section{Méthodologie expérimentale en traitement du signal audionumérique (5)}

y=f'(x)

recherche reproductible, explanes 5

etude de f' par la construction d'un x'

definition de y

\section{La synthèse au service de l'écoute artificielle (5)}

Dcase

\section{La synthèse au service de la psychologie expérimentale (5)}

Mcgill, these de Grégoire
