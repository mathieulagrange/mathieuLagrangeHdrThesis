\begin{frame}{Analyse de Scènes Auditives Computationnelle (CASA)}
L'Analyse de Scène Auditives (ASA)\footfullcitenomarkleft{bregman1994auditory} étudie l'ensemble de traitements perceptifs permettant
\begin{itemize}
\item d'isoler les informations émanant d'\structure{entités} sonores distinctes,
\item de les organiser en un tout cohérent,
\item à l'aide de processus \structure{\og primitifs \fg} et \structure{\og séquentiels \fg}.
\end{itemize}
L'approche CASA\footfullcitenomarkleft{wang2016casa} met en \oe{}uvre ces critères pour inférer \structure{automatiquement} une organisation perceptuellement valide de la scène sonore.
\end{frame}

\begin{frame}{Processus ASA \og primitifs \fg}
\begin{description}
\item[\alert<2>{continuité}] : les propriétés d'un son isolé tendent à se modifier lentement et de façon continue
\item[harmonicité] : lorsqu'un corps sonore vibre à une période répétée, ses vibrations donnent naissance à un motif acoustique dont les fréquences des composants sont des multiples d'une même fréquence fondamentale;
\item[...]
\end{description}
\end{frame}

\begin{frame}{Modèle sinusoïdal à long terme}
\only<1,2>{
$$
x[n]=\sum_{l=1}^{L} a_{l}[n] \sin \left(\frac{2 \pi}{F_{s}} f_{l}[n] \cdot n + \Phi_k \right)
$$
}
\only<1>{
$a_{l}[n]$ et $f_{l}[n]$ sont des signaux \alert{basse fréquence} contrôlant respectivement l'amplitude et la phase des $L$ oscillateurs composant le modèle.
}
\only<2>{
\begin{itemize}
  \item $a_{l}[n]=1$
  \item $f_{l}[n]=l*\left(f_0+a_v \sin \left(\frac{2 \pi * 5}{F_{s}} \cdot n \right)+FPB(\mathcal{N}) \right)$
  \item $f_0$: fréquence de fondamentale
  \item $FBP$: filtre passe bas
\end{itemize}

}
\begin{center}
\only<3>{\includegraphics[width=.9\columnwidth]{figures/asaSpec} \\
 John Chowning (1970-80) \\
 \includesound{sounds/asa.mp3}}
\end{center}
%http://webpages.mcgill.ca/staff/Group2/abregm1/web/snd/Track24.mp3
\end{frame}

\begin{frame}{Procédé d'analyse de signaux sonores}
\begin{center}
\begin{tikzpicture}[mystyle]

  \matrix [column sep=10mm,row sep=5mm]
{
\node (i1) {}; \\
\node [terminal] (i2) {tfct}; \\
\node [terminal] (i3) {sélection de pics}; \\
\node [terminal] (i4) {suivi de partiel}; \\
\node  (i5) {}; \\
};

\begin{scope}[every path/.style=line]
  \path (i1) -- node [right] {signal} (i2);
  \path (i2) -- node [right] {spectrogramme} (i3);
  \path (i3) -- node [right] {atomes } (i4);
  \path (i4) -- node [right] {partiels} (i5);
\end{scope}
\end{tikzpicture}
\end{center}
\end{frame}

\begin{frame}{Atomes temps/fréquences}
\begin{center}
 \only<1>{\includegraphics[width=.6\textwidth]{voice_1024_512xp} \\
 Taille de fenêtre : \structure{23} ms, pas d'avancement 10 ms.}
 \only<2>{\includegraphics[width=.6\textwidth]{voice_2048_512xp} \\
 Taille de fenêtre : \structure{46} ms, pas d'avancement 10 ms.}
 \only<3>{\includegraphics[width=.6\textwidth]{voice_4096_512xp} \\
 Taille de fenêtre : \structure{92} ms, pas d'avancement 10 ms.}
\end{center}
\end{frame}

\begin{frame}{Algorithme de suivi amélioré\footfullcitenomarkleft{lagrangeTaslp06}}
\begin{center}
  \includegraphics[width=.35\textwidth]{trackingxp2xp2}
\end{center}
  \begin{enumerate}
      \item Prédiction auto-régressive de l'évolution des paramètres de fréquence et d'amplitude
      \item Sélection des continuations engendrant le moins de hautes fréquences
  \end{enumerate}
\end{frame}

%todo minipage

\begin{frame}{Processus ASA \og primitifs \fg}
\begin{description}
\item[\alert{continuité}] : les propriétés d'un son isolé tendent à se modifier lentement et de façon continue
\item[\alert{harmonicité}] : lorsqu'un corps sonore vibre à une période répétée, ses vibrations donnent naissance à un motif acoustique dont les fréquences des composants sont des multiples d'une même fréquence fondamentale;
\item[...]
\end{description}
\end{frame}

\begin{frame}{Algorithme par coupures normalisées de graphes}
\begin{center}
 \only<1>{\includegraphics[width=.7\textwidth]{orcaSin}}
 \only<2>{\includegraphics[width=.8\textwidth]{figures/ncutDiagramFr} \footfullcitenomarkleft{shi2000normalized}}
 \only<3>{\includegraphics[width=.7\textwidth]{orcaSep}  \footfullcitenomarkleft{lagrangeTaslp08}}
\end{center}
\end{frame}



\begin{frame}{Processus ASA \og séquentiels \fg}
\begin{description}
\item[proximité] : des éléments proches les uns des autres sur le plan temps/fréquence ont tendance à être groupés ensemble
\item[similarité] : des éléments qui se ressemblent ont tendance à être groupés ensemble (timbre).
\item[...]
\end{description}
\end{frame}

\begin{frame}{Regroupement hiérarchique alterné (ANR JCJC Houle)}
\footfullcitenomarkleft{rossignol2018efficient}
\footfullcitenomarkleft{rossignolhal-01122006}
\begin{center}
   \includegraphics[width=.9\textwidth]{alc_sample}
\end{center}
\end{frame}

\begin{frame}{Approches \og algorithmiques \fg}
\begin{itemize}
\item Expression d'\textit{a priori} sous forme d'heuristiques computationnelles
\item Algorithmes de structuration non supervisés
\end{itemize}
\begin{block}{Bilan}
\begin{itemize}
\item[\textbf{+}] approches long terme : exploitation d'intervalles d'observation long
\item[\alert{\textbf{-}}] représentation court terme : problèmes de résolution
\item[\textbf{+}] pas d'apprentissage : protocole expérimental simple, interprétabilité
\item[\alert{\textbf{-}}] pas d'apprentissage : problèmes de tractabilité, problèmes d'efficience
\end{itemize}
\end{block}
\end{frame}
