\chapter{\nmu Notice \nmu de lecture} \label{chap:notice}

Ce document, divisée en trois parties, présente de manière synthétique mes contributions, résultats de 17 années de recherche en modélisation du signal sonore. Ces parties peuvent être lues en relative indépendamment.

La première partie présente dans une chronologie globalement respectée, les contributions principales de mon effort de recherche dans le domaine de \textbf{l'analyse computationnelle de scènes sonores}. Un premier temps s'est opéré frontalement, en postulant une formulation canonique du problème. Par un constat d'insuffisance et l'identification des conditions nécessaires à une approche plus rigoureuse, j'ai orienté ma recherche autour de deux contributions :
\begin{enumerate}
  \item Comment mieux \textbf{contrôler les données d'expérimentation} pour augmenter la valeur qualitative des résultats expérimentaux ?
  \item Comment les outils de manipulation numérique du signal sonore permettent de mieux \textbf{questionner l'impact sur la perception humaine de l'exposition à des stimuli sonores} ?
\end{enumerate}.

La seconde partie est dédie à une présentation approfondie de la \textbf{modélisation long terme du signal sonore}. J'y dresse un panorama de l'évolution de l'effort de recherche déployé par la communauté dans ce domaine durant ces 20 dernières années en détaillant certaines contributions que j'ai pu apporter.

Enfin, je termine par une réflexion sur l'usage moderne de la méthode scientifique en sciences des données. J'y présente le fruit de mes investigations dans la formalisation et la conception de protocole expérimentaux destinés à faciliter la recherche reproductible dans ces domaines nouveaux.

  %La première est une présentation synthétique de mon parcours académique, de ma thèse de doctorat débutée en 2001 à France Télécom R\&D à mon intégration en tant que chargé de recherche CNRS au laboratoire des sciences du numérique de Nantes (LS$2$N UMR $6004$).
